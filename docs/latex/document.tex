\documentclass{report}

% bibliography
\usepackage[
style=authoryear,
backend=biber,
]{biblatex}
\addbibresource{bibliography.bib}

\title{Research Project Documentation}
\author{Conor Olive}
\date{2023}

% page setup
\usepackage[
top=1.2in,
bottom=1.2in,
left=1.65in,
right=1.65in,
]{geometry}

% color
\usepackage[usenames,dvipsnames,table]{xcolor}

% AMS packages
\usepackage{amsmath, amsfonts, amsthm, amssymb}

\renewcommand{\implies}{\rightarrow}
\renewcommand{\iff}{\leftrightarrow}
\newcommand{\xor}{\oplus}

% links
\usepackage{hyperref}
\hypersetup{
	colorlinks,
	linkcolor={black},
	citecolor={black},
	urlcolor={blue!80!black}
}
\usepackage{float}
\usepackage{import}
\usepackage{xifthen}
\usepackage{pdfpages}
\usepackage{transparent}

% Code snippets
\usepackage{listings}

\definecolor{codegreen}{rgb}{0,0.6,0}
\definecolor{codegray}{rgb}{0.5,0.5,0.5}
\definecolor{codepurple}{rgb}{0.58,0,0.82}
\definecolor{backcolour}{rgb}{1,1,1}

\lstdefinestyle{lststyle}{
	language=Python,
	backgroundcolor=\color{backcolour},   
	commentstyle=\color{codegreen},
	keywordstyle=\color{blue},
	numberstyle=\color{codegray},
	stringstyle=\color{codepurple},
	basicstyle=\ttfamily,
	breaklines=true,                 
	%numbers=left,                    
	tabsize=2,
	frame=single,
}

\lstset{style=lststyle}

% fancy headers
%\usepackage{fancyhdr}
%\pagestyle{fancy}
%
%\fancyhead[LE,RO]{Conor Olive}
%\fancyhead[RO,LE]{}
%\fancyhead[RE,LO]{}
%\fancyfoot[LE,RO]{\thepage}
%\fancyfoot[C]{\leftmark}

% inkscape figure inclusion 
\newcommand{\incfig}[1]{%
	\def\svgwidth{\columnwidth}
	\import{./figures/}{#1.pdf_tex}
}

% theorem environments
\makeatother
\usepackage{thmtools}
\usepackage[framemethod=TikZ]{mdframed}
\mdfsetup{skipabove=1em,skipbelow=0em}

\theoremstyle{definition}

\declaretheoremstyle[
headfont=\bfseries\sffamily\color{ForestGreen!70!black}, bodyfont=\normalfont,
mdframed={
	linewidth=2pt,
	rightline=false, topline=false, bottomline=false,
	linecolor=ForestGreen, backgroundcolor=ForestGreen!5,
}
]{thmgreenbox}

\declaretheoremstyle[
headfont=\bfseries\sffamily\color{NavyBlue!70!black}, bodyfont=\normalfont,
mdframed={
	linewidth=2pt,
	rightline=false, topline=false, bottomline=false,
	linecolor=NavyBlue, backgroundcolor=NavyBlue!5,
}
]{thmbluebox}


\declaretheoremstyle[
headfont=\bfseries\sffamily\color{NavyBlue!70!black}, bodyfont=\normalfont,
mdframed={
	linewidth=2pt,
	rightline=false, topline=false, bottomline=false,
	linecolor=NavyBlue
}
]{thmblueline}

\declaretheoremstyle[
headfont=\bfseries\sffamily\color{RawSienna!70!black}, bodyfont=\normalfont,
mdframed={
	linewidth=2pt,
	rightline=false, topline=false, bottomline=false,
	linecolor=RawSienna, backgroundcolor=RawSienna!5,
}
]{thmredbox}

\declaretheoremstyle[
headfont=\bfseries\sffamily\color{RawSienna!70!black}, bodyfont=\normalfont,
numbered=no,
mdframed={
	linewidth=2pt,
	rightline=false, topline=false, bottomline=false,
	linecolor=RawSienna, backgroundcolor=RawSienna!1,
},
qed=\qedsymbol
]{thmproofbox}

\declaretheoremstyle[
headfont=\bfseries\sffamily\color{NavyBlue!70!black}, bodyfont=\normalfont,
numbered=no,
mdframed={
	linewidth=2pt,
	rightline=false, topline=false, bottomline=false,
	linecolor=NavyBlue, backgroundcolor=NavyBlue!1,
},
]{thmexplanationbox}


\declaretheorem[style=thmbluebox, name=Algorithm]{algorithm}
\declaretheorem[style=thmgreenbox, name=Definition]{definition}
\declaretheorem[style=thmbluebox, numbered=no, name=Example]{eg}
\declaretheorem[style=thmbluebox, numbered=no, name=Solution]{solution}
\declaretheorem[style=thmredbox, name=Proposition]{prop}
\declaretheorem[style=thmredbox, name=Theorem]{theorem}
\declaretheorem[style=thmredbox, name=Lemma]{lemma}
\declaretheorem[style=thmredbox, numbered=no, name=Corollary]{corollary}
\declaretheorem[style=thmproofbox, name=Proof]{replacementproof}
\renewenvironment{proof}[1][\proofname]{\vspace{-10pt}\begin{replacementproof}}{\end{replacementproof}}
\declaretheorem[style=thmexplanationbox, name=Proof]{tmpexplanation}
\newenvironment{explanation}[1][]{\vspace{-10pt}\begin{tmpexplanation}}{\end{tmpexplanation}}
\declaretheorem[style=thmblueline, numbered=no, name=Remark]{remark}
\declaretheorem[style=thmblueline, numbered=no, name=Note]{note}
\newtheorem*{uovt}{UOVT}
\newtheorem*{notation}{Notation}
\newtheorem*{previouslyseen}{As previously seen}
\newtheorem*{problem}{Problem}
\newtheorem*{observe}{Observe}
\newtheorem*{property}{Property}
\newtheorem*{intuition}{Intuition}

%\setlength{\arrayrulewidth}{0.5mm}
%\setlength{\tabcolsep}{18pt}

\renewcommand{\arraystretch}{1.2}
{\rowcolors{3}{NavyBlue!2}{NavyBlue!2}
	\setlength{\arrayrulewidth}{0.25mm}
	\arrayrulecolor{NavyBlue!100}



\begin{document}
	\tableofcontents
	
	\chapter{Gravity Currents}
	
	\section{Two-Dimensional Box Model with Source}
	Begin with the assumption that the volume continuity is not constant, that is 
	\begin{equation}\label{eq:source-term}
		x_N(t)h_N(t) = V = qt^{\alpha}\text{,}
	\end{equation}
	where \(q\) is a constant and \(\alpha\) is a constant greater than zero. Next, consider the ordinary differential equation (see \cite[60]{Ungarish2009}), 
	\begin{equation}\label{eq:box-model}
		\frac{dx_N(t)}{dt} = \text{Fr}\left[h_N(t)\right]^{1/2}\text{.}
	\end{equation}
	Here, we also treat the Froude number as a constant. The height function \(h_N(t)\) can be substituted for \(qt^{\alpha} / x_N(t)\), from \eqref{eq:source-term} as in
	\begin{equation*}
		\frac{dx_N(t)}{dt} = \text{Fr}\left[\frac{qt^{\alpha}}{x_N(t)}\right]^{1/2}\text{.}
	\end{equation*}
	This can then be solved for \(x\) as a seperable differential equation,
	\begin{equation*}
		\frac{1}{\text{Fr}}\int_{x_0}^{\bar{x}} \left[x_N(t)\right]^{1/2} dx_N = q^{1/2} \int_{t_0}^{\bar{t}} t^{\alpha/2} dt\text{,}
	\end{equation*}
	which when integrated yields
	\begin{equation*}
		\frac{2}{3}\text{Fr}^{-1}\left[x_N(t) - x_0\right]^{3/2} = q^{1/2}\frac{t^{\frac{\alpha}{2} +1}}{\frac{\alpha}{2} + 1}\text{.}
	\end{equation*}
	Finally, solving for \(x_N(t)\) yields
	\begin{equation*}
		x_N(t) = \left[\text{Fr}\cdot\frac{3q^{1/2}t^{3\alpha/2 }}{2\alpha + 4} + x_0^{3/2}\right]^{2/3}\text{.}
	\end{equation*}

	\chapter{Numerical Methods}

	\section{Runge-Kutta Methods}
	\subsection{General Form of Runge-Kutta Methods}
	
	From \cite[134]{Hairer2008}
	
	\begin{definition}[Explicit Runge-Kutta Method]
		Let \(s\) be an integer (the "number of stages") and \(a_{21}, a_{31}, a_{32}, \dots, a_{s1}, a_{s2},\dots,a_{s,s-1}\), \(b_1, \dots, b_s\), \(c_2, \dots,c_s\), be real coefficients. Then the method
		\begin{align}
			\begin{split}
			k_1 &= f(x_0, y_0)\\
			k_2 &= f(x_0 + c_2h , y_0 + ha_{21}k_1)\\
			k_3 &= f(x_0 + c_3h, y_0 + h(a_{31}k_1 + a_{32}k_2))\\
			\: &\dots \\
			k_s &= f(x_0 + c_s h, y_0 + h(a_{s1}k_1 + \dots + a_{s,s-1}k_{s-1}))\\
			y_1 &= y_0 + h(b_1k_1 + \dots + b_sk_s)
			\end{split}
		\end{align}
		is called an \textit{\(s\)-stage explicit Runge-Kutta method} (ERK).
	\end{definition}
	\vspace{1cm}
	\noindent Most often these methods can be expressed as a Butcher tableau, e.g.
	\begin{table}[H]
		\centering
		\begin{tabular}{c|ccccc}
			  \(0\)    &            &            &            &               &         \\
			 \(c_2\)   & \(a_{21}\) &            &            &               &         \\
			 \(c_3\)   & \(a_{31}\) & \(a_{32}\) &            &               &         \\
			\(\vdots\) & \(\vdots\) & \(\vdots\) & \(\vdots\) &               &         \\
			 \(c_s\)   & \(a_{s1}\) & \(a_{s2}\) & \(\dots\)  & \(a_{s,s-1}\) &         \\ \hline
			           &  \(b_1\)   &  \(b_2\)   & \(\dots\)  &  \(b_{s-1}\)  & \(b_s\)
		\end{tabular}
	\end{table}
	
	\subsection{Automatic Step Size Control}
	
	
	
	\printbibliography

\end{document}